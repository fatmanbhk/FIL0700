\documentclass[11pt]{article}
\usepackage[utf8]{inputenc}
\usepackage{listings}
\usepackage{graphicx}
%Gummi|065|=)
\title{\textbf{FIL 0700 - Seminar 20.09.17}}
\author{Bjørn Karlsen}
\date{}
\begin{document}

\maketitle

\section{Lekser: Kap. 2.2.5 s.80-94}

\subsection{Begrep}
\begin{itemize}
\item{Fundasjonalisme}
\item{Rasjonalisme}
\item{Empirisme}
\end{itemize}

Analytisk/ syntetisk (apriori/ aposteriori)

\section{FRIHET OG MORALSK ANSVAR}

\subsection{Kriterier for å kunne sies å ha moralsk ansvar}
\begin{itemize}
\item{Handlingsfrihet}
\item{Viljesfrihet}
\item{Tilstrekkelig kunnskap}
\end{itemize}

\subsection{Handlingsfrihet}
Ønsker/ besemt seg for det.
En katt har f.eks handlingsfrihet. Knuser den en vase ved et uhell, så har ikke den et moralsk ansvar for det, da den ikke har noen forståelse for hva den har gjort/ ikke ønsket å gjøre det.
(dette er \emph{IKKE} tilstrekkelig for at man kan si at man har moralsk ansvar)
EKS: de som lagde ovnene som skulle brenne jøder under 2.verdenskrig, hadde nødvendigvis ikke moralsk ansvar fordi de bygget de under press. Man kan si at de \emph{ikke} hadde moralsk ansvar fordi de ikke hadde handlingsfrihet.

\subsection{Fri vilje (\emph{Viljesfrihet)}}
\emph{EKS:} \emph{Ønsker} å gå på do, men kan bestemme seg for å utsette det hvis man vil. Et annet eksempel er at man kan \emph{bestemme} seg for å legge seg tidliger på kvelden slik at man ikke \emph{ønsker} å sove mer neste dag.
Et annet eks. igjen kan være at man bestemmer seg for å ikke ha Facebook-konto slik at man slipper å ta stilling til alle ønskene til de andre på Facebook (\emph{prioritere}).
Har man ikke fri vilje til å bestemme f.eks. om man ikke vil stjele (kleptomani/ tvangstanker), man vet at det er galt men klarer ikke å få seg til å la være, så har man da heller ikke moralsk ansvar for at man f.eks. "stjal" den klokken.


\subsection{Handlingsfrihet og Viljesfrihet \emph{uten} moralsk ansvar}
\emph{EKS:} en kokk lager en pizza og noen forgifter den uten at kokken vet det. Han serverer den og den som spiser den dør. Han har da ikke moralsk ansvar pga. \emph{manglende kunnskap}.

\section{Radikal frihet (Sartre - Eksistensialisme)}
"Mennesket er intet annet enn det det gjør seg selv til".
Vi blir til gjennom våre valg og handlinger, og på den måten skaper vi den vi er. Og i og med at vi er ansvarlige for våre handlinger, er vi også ansvarlige for den vi blir til og er.

\section{Induskjonisme}
Observasjon: svaner er hvite -> Teori: alle svaner er hvite\\
Dersom noe er sant om \emph{noen} av svanene, så da går vi ut fra at det også er sant om \emph{alle} svanene (fra "en/ noen" til "alle").
Logisk gyldighet: dersom observasjonene er gyldige, så \emph{må} konklusjonen også være det.

\section{Deduksjonisme}
Vil si at vi sier at om noe er sant om \emph{alle} svanene så må det også være sant om \emph{noen} av svanene (fra "alle" til "en/ noen").


\end{document}