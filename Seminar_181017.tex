\documentclass[11pt]{article}
\usepackage[utf8]{inputenc}
\usepackage{listings}
\usepackage{graphicx}
%Gummi|065|=)
\usepackage[T1]{fontenc}
\title{\textbf{FIL 0700 - Seminar 18.10.17}}
\author{Bjørn Karlsen}
\date{}
\begin{document}

\maketitle

\section{Les 4.1 - 4.3 (side 151-167)}
Lekser - ark. Rep. (1),(2), side 194

\section{Begrep}
Fornuft -> visdom -> ledelse\\
Vilje -> mot -> vokter\\
Begjær(karakteregenskap) -> måtehold(dyd) -> produsenter\\\\
Kant's Universaliseringsprinsipp:\\
"Handle bare etter den maksime gjennom hvilken du samtidig kan ville at den skal bli en allmenn lov". Betyr handlingsregel. Prinsippet krever at vi ikke skal gi oss selv handlingsregler som vi ikke samtidig vil skal gjelde for alle andre. Det katergoriske imperativet fungerer som en test som skal brukes for å avgjøre om en handlingsregel er moralsk riktig eller ikke. Moralloven = det kategoriske imperativ.\\\\
Humanitetsprinsippet:\\
Sier noe om hvordan vi skal betrakte andre mennesker.\\ "Handle slik at du bruker menneskeheten både i din egen person og i enhver annen person samtidig som formål og ikke bare som middel"
\subsection{}
\begin{itemize}
\item{}
\item{}
\item{}
\end{itemize}

\subsection{}

\subsection{}

\subsection{}
\section{}

\section{}

\section{}



\end{document}