\documentclass[11pt]{article}
\usepackage[utf8]{inputenc}
\usepackage{listings}
\usepackage{graphicx}
%Gummi|065|=)
\title{\textbf{FIL 0700 - 1.6.1, 1.6.2}}
\author{Bjørn Karlsen}
\date{}
\begin{document}

\maketitle

\section{Seminar 13.09.17}

\subsection{Determinisme}
Ikke fri vilje.
Enhver tilstand og hendelse er bestemt av forutgående betingelser.

\subsection{Indeterminisme}
Fri vilje.
Noen hendelser er ikke determinert.

\subsection{Handlingsfrihet}
Muligheten til å \emph{handle} utfra egne ønsker.

\subsection{Fri vilje}
Muligheten til \emph{kontroll} over egne ønsker.

\section{Oppgaveark}

\subsection{Gi et eksempel på en logisk gyldig slutning. Hva er premissene?}

P1) Alle mennesker er dødelige.

P2) Sokrates er et menneske.

K)  Sokrates er dødelig.

Logisk gyldig: \emph{UMULIG} at premissene er sanne og konklusjonen usann.

\emph{HVIS} premissene er sanne, så \emph{MÅ} konklusjonen være gyldig.

\subsection{Gi et eksempel på induktiv generalisering. Hva er premissene? Hva er konklusjonen? Gi et moteksempel som bevis på at ditt eksempel ikke er gyldig. Hvorfor er sånne slutninger uansett gyldige?}

\subsection{Eksempler på ressonementer som ikke er logisk "gyldige", men helt rimelig.}
\begin{itemize}
\item{Induksjon}
\item{Analogi-slutning}
\item{Abduksjon}
\end{itemize}

\subsection{Hva er "analogier" og "metaforer?" Gi eksempler.}
Elgen er skogens konge.

\subsection{Analogi-slutning}
Man antar at det finnes andre likheter basert på at det finnes \emph{EN} likhet.

\begin{itemize}
\item{Induksjon}
\item{Analogi-slutning}
\item{Abduksjon}
\end{itemize}

\section{Neste gang:}
\begin{itemize}
\item{Commute well-being}
\item{Les s.224 i boka}
\item{Utdrag: Semmelweis, HDM, komplikasjoner}

\end{itemize}

\end{document}