\documentclass[11pt]{article}
\usepackage[utf8]{inputenc}
\usepackage{listings}
\usepackage{graphicx}
\usepackage{amsmath}
%Gummi|065|=)
\title{\textbf{ExPhil forelesning 29.09.17}}
\author{Bjørn Karlsen}
\date{}
\begin{document}

\maketitle

\section{Kap 3.Heinz' dilemma}
To ulike moralske stemmer
Omsorgsetikken: "Det andre" synet må ikke nedvurderes bare fordi det mer preger kvinners tenking
gutter: rettferdighets-\\tenkning som betoner universelle prinsipper\\
jenter: omsorgstenking som betoner komplekse, konteksuelle sosiale relasjoner og følelser.

\section{Kap.4 - Politikk og samfunn}
Tilnærmingsmåter til det politiske:
Deskriptiv politisk teori
statsvitenskap, sosiologi, soialantropologi
Hvorfor ER samfunnet organisert?

Normativ politisk teori
politisk filosofi
Hvorfor BØR samfunnet organiseres?

Sentale spm.:
hvordan bør samfunnet orrganiseres?
Hvordan organisere oss på en rettferdig måte?
Hva er politisk og sosial retferdighet?
Hvordan organisere oss på en stabil og varig måte?

Hva innebærer begreper som:
sikkerhet og trygghet
frihet
rettsstat
demokrati
\end{document}