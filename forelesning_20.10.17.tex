\documentclass[11pt]{article}
\usepackage{listings}
\usepackage{color}
\definecolor{mygreen}{rgb}{0,0.6,0}
\definecolor{mygray}{rgb}{0.5,0.5,0.5}
\definecolor{mymauve}{rgb}{0.58,0,0.82}
\definecolor{mycolor}{rgb}{0.99,0.99,0.70}
\definecolor{myblack}{rgb}{0.55,0.55,0.55}
\usepackage[utf8]{inputenc}
\usepackage{graphicx}
\usepackage{amsmath}
%\usepackage[T1]{fontenc}
\lstset{ %
backgroundcolor=\color{mycolor},
basicstyle=\footnotesize,
breakatwhitespace=false,
breaklines=true,              
captionpos=b,                    
commentstyle=\color{mygreen},   
deletekeywords={...},           
escapeinside={\%*}{*)},         
extendedchars=true,            
frame=shadowbox,                  
keepspaces=true,                
keywordstyle=\color{blue},       
language=SQL,                
morekeywords={*,...},           
numbers=left,                  
numbersep=5pt,                   
numberstyle=\tiny\color{mygray},
rulecolor=\color{black},
rulesepcolor=\color{myblack},    
showspaces=false,             
showstringspaces=false,         
showtabs=false,                
stepnumber=2,                   
stringstyle=\color{mymauve},
tabsize=2,	                  
title=\lstname
}
\title{\textbf{FIL 0700\linebreak[2.0]\\Forelesning 20.10.17}}
\author{Bjørn Karlsen}
\date{18.10.17}
\begin{document}

\maketitle

\section{Hvordan får vi kunnskap?}
DESCARTES (1596- 1650):\\
Rasjonalisme - apriori fundament\\\\
Metodisk tvil:\\
- mål: finne et absolutt urokkelig og sikkert utgangspunkt for all kunnskap\\
- bevise at man kan ha kunnskap om verden\\
- METODE: midlertidig forkaste alt en kan tvile det minste på for å se om noe viser seg ubetvilelig.\\\\
- sansene kan betviles:\\
- illusjoner, hallusinasjoner\\
- forkaster mye aposteriori begrunnelse\\\\
- all daglig erfaring kan betviles\\
- alt kan være en livakig drøm\\
- forkaster all aposteriori begrunnelse\\\\
- Matematikk og logiske slutninger kan betviles\\
- En ond demon kan tenkes å gjøre alt for å lure meg til usannhet og forstyrre meg når jeg konkluderer.\\
- Forkaster mye apriori begrunnelse.\\\\
Finnes det så noe i det hele tatt som ikke kan betviles og som kan stå som fundament?\\
- Ja fordi jeg kan ikke betvile min egen eksistens\\
- Jo mer jeg tviler, jo mer tenker jeg, og desto sikrere er det at jeg må være til noe som tenker og tviler\\
- cogito ergos sum (jeg tenker derfor er jeg)\\

Descartes har også en klar/ tydelig forestilling om et fullkomment vesen\\
Selv er han ikke fullkommen fordi da hadde han ikke slitt med å finne kunnskap\\
Så han selv, et ufullkomment vesen, kan ikke være opphavet til forestillingen om et fullkomment vesen fordi en virkning (hans forestilling) kan ikke ha
Er det noe feil med dette argumentet? JA








\end{document}